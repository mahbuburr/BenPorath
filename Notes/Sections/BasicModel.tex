\section{Basic Ben-Porath Model} \label{section:baseline}
\begin{assumption} (Basic Ben-Porath Model) \label{assumption:bbpmodel}
The assumptions of the basic Ben-Porath model are: (i) homogeneity (single representative agent); (ii) perfect capital markets; (iii) no non-market benefits from human capital; (iv) fixed labor supply; (v) constant depreciation of human capital, $\sigma$; (vi) finite horizon.
\end{assumption}

\noindent Let Assumption~\ref{assumption:bbpmodel} hold for the rest of this section. For each $t \in [0,T]$, $H$ denotes human capital, $I \in [0,1]$ allocation to investment in human capital, $D$ market goods, and $F(I,D)$ the production function of human capital stock. Thus, the human capital stock is produced through two inputs, $D$ and $I$. 

\begin{assumption} (Strict Concavity of the Production Function) \label{assumption:scpf}  $\forall \ t \ in [0,T] \ F(\cdot, \cdot)$ is strictly concave in both of its arguments.
\end{assumption}

\begin{definition} (Law of Motion for Human Capital Stock in the Basic Ben-Porath Model)
\ldots is defined as
\begin{equation}
\dot{H(t)} = F \left( I(t) H(t), D(t) \right) - \sigma H(t). \label{eq:lawh}
\end{equation}
\end{definition}

\begin{remark} (Neutrality)
In the basic Ben-Porath model the law of motion for human capital stock embeds a neutrality assumption. Namely, the current stock of human capital at time $t$, $H(t)$, and the investment time at time $t$, $I(t)$, appear as a single argument in a multiplicative fashion in the flow production of human capital stock. 
\end{remark}

\indent At each point of time, the current stock and the rental rate of human capital, $R$, define potential earnings as $Y(t) = R H(t)$. In general, earnings and potential earnings differ by two terms: (i) foregone earnings; (ii) direct market goods costs. 

\begin{definition} (Earnings) \label{definition:earnings}
Let $P_{D}$ be the price of markets. Earnings are
\begin{equation}
E(t) = R H(t) -  R I(t) H(t) - P_{D} D(t) \label{eq:earnings}
\end{equation}

\noindent where $R I(t) H(t)$ are foregone earnings and $ P_{D} D(t) $ are direct goods costs. 
\end{definition}

\indent Definition~\ref{definition:earnings} clarifies that $I(t)$ is the allocation to investment in each period of time. In particular, the individual occupies a fraction $I(t)$ of her human capital stock to produce human capital. The individual chooses $D(t)$ and $I(t)$ to maximize her lifetime earnings stream given an initial level of human capital, $H(0) = H_{0}$ and subject to the law of motion for human capital, \eqref{eq:lawh}. Explicitly,

\begin{problem} \label{problem:bbp} (Life-cycle Individual's Problem in the Basic Ben-Porath Model)
\begin{equation}
\max_{I_{t}, D_{t}} \int \limits _{0} ^{T} \exp^{-rt} RH(t)(1 - I(t)) dt \nonumber \\
\end{equation}
\noindent s.t.
\begin{eqnarray}
H(0) = H_{0} \nonumber \\
\text{\eqref{eq:lawh} holds} \nonumber .
\end{eqnarray}
\end{problem}

\indent The \textbf{current value Hamiltonian} associated to Problem \ref{problem:bbp} is
\begin{equation}
\mathcal{H} (\cdot) = \exp^{-rt} \left[ R H(t) -  R I(t) H(t) - P_{D} D(t) \right] + \mu(t) \dot{H(t)} 
\end{equation}

\noindent where $\mu(t)$ defines the shadow price of the human capital stock. Assumption~\ref{assumption:bbpmodel} guarantees that the necessary and sufficient conditions for optimality are the following.
\begin{condition} (Optimality Conditions for the Life-cycle Individual's Problem in the Basic Ben-Porath Model) 
\begin{eqnarray}
\frac{\partial \mathcal{H} (\cdot)}{\partial I(t)} = 0 &\Leftrightarrow& \exp^{-rt}R = \mu(t) F_{I(t)H(t)} \label{eq:focinvestment} \\
\frac{\partial \mathcal{H} (\cdot)}{\partial D(t)} = 0 &\Leftrightarrow& \exp^{-rt}R I(t) = \mu(t) F_{D(t)} \label{eq:focgoods} \\
\frac{\partial \mathcal{H} (\cdot)}{\partial H(t)} = - \dot{\mu(t)} &\Leftrightarrow& \exp^{-rt} R \left( 1 - I (t) \right) + \mu(t) \left(  F_{I(t)H(t)} - \sigma \right) = - \dot{\mu(t)} \label{eq:focstock} \\ 
\frac{\partial \mathcal{H} (\cdot)}{\partial \mu(t)} = \dot{H(t)} &\Leftrightarrow& \dot{H(t)} = F \left( I(t) H(t), D(t) \right) - \sigma H(t) \label{eq:focmotion} \\
\text{Transversality} &:& \lim_{t \rightarrow T} \mu(t) H(t) = 0 \label{eq:foctransversality}
\end{eqnarray}
where $F_{j} \equiv \frac{\partial F \left( I(t), D(t) \right) }{\partial j}$ for $j = D(t), I(t) H(t)$.
\end{condition}

\indent In order to analyze some aspects of this model, it is useful to solve the period-by-period counterpart of Problem~\ref{problem:bbp}. We can think of this as a problem in which, each period, the agent maximizes gross investment in human capital less input costs. Provided that we have the adequate discount factor we can write the period-by-period counterpart. $ \forall \ t \in [0,T]$ define $g(t)$ as a discount factor and write the period-by-period problem.
\begin{problem} (Period-by-Period Individual's Problem in the Baisc Ben-Porath Model)\label{problem:perbyperbbpm}
\begin{eqnarray}
\max_{I(t),D(t)} \left[ g(t) F \left( I(t),H(t) \right) - P_{D}D(t) - RI(t)H(t) \right]. \nonumber
\end{eqnarray}
\end{problem} 

\indent Actually, we can interpret Problem~\ref{problem:perbyperbbpm} as a production problem: the individual is a firm that receives $g(t)$ (return to gross investment in human capital) for the production of human capital investment through the technology $F \left( I(t),H(t) \right)$. It pays the prices $P_{D}, R$ for the inputs $D(t), I(t)H(t)$.

\begin{claim} (Life-cycle and Period-by-period Solution Equivalence) 
Let $g(t) \equiv \exp^{rt} \mu(t)$. Then, the solution to Problem~\ref{problem:bbp} and Problem~\ref{problem:bbp} are equivalent.
\end{claim}

\begin{proof}
The equivalence follows after comparing the first order conditions (note that $g(t)$ makes equivalent the first order conditions of the two problems).
\end{proof}

\indent Economically, $g(t)$ is a discount factor that adjusts for the correct exponential depreciation of gross investment so that the period-by-period and the life-cycle solutions coincide. In order to analyze the solution combine \eqref{eq:focinvestment} and \eqref{eq:focstock} to get 
\begin{equation}
\dot{\mu(t)} = - \exp^{-rt} R + \mu(t) \sigma \label{eq:focinvstockcombine}
\end{equation}

\noindent and note that $\dot{g(t)} = \dot{\mu(t)} \exp^{rt} + r \mu(t) e ^{rt}$. Use \eqref{eq:focinvstockcombine} to obtain 
\begin{equation}
\dot{g(t)} = (\sigma + r ) g(t) - R. \label{eq:grossdep}
\end{equation}

\indent Equation~\eqref{eq:foctransversality} implies that $\mu(T) = 0 $ and, therefore, $g(T) = 0$ provided that $H(t) = 0$ because conditions under which $H(T) = 0$ have no economic sense. It is possible, thus, to solve \eqref{eq:grossdep} and obtain
\begin{equation}
g(t) = \frac{R}{\sigma + r} \left[ 1 - \exp^{(\sigma + r)(t - T)} \right].
\end{equation}

\noindent which leads to $\dot{g(t)} < 0$. To wrap up the discussion note that the optimality conditions for Problem \ref{problem:perbyperbbpm} are the following.\\
 
\begin{condition} (Optimality Conditions for the Period-by-period Individual's Problem in the Basic Ben-Porath Model)
\begin{eqnarray}
g(t) F_{I(t)H(t)} H(t) &=& R H (t) \nonumber \\
g(t) F_{D(t)} H(t) &=& P_{D} \label{eq:newfocs}. 
\end{eqnarray}
\end{condition}

\indent The system in \eqref{eq:newfocs} consists of two equations and two unknowns that solve for the Marshallian demands for $I(t)H(t)$ and $D(t)$. Assumption~\ref{assumption:scpf} together with $\dot{g(t)} < 0$ imply that the both Marshallian demands are decreasing overtime, which is intuitive because the agent faces a finite horizon problem.

\subsection{Earnings Dynamics}
\indent One of the fundamental questions that this basic model enables to ask is how earnings evolve over the life-cycle. Consider Claim \ref{claim:earnnodep} and Claim \ref{claim:earndep}

\begin{claim} (Earnings over Time with no Depreciation) \label{claim:earnnodep}
Let $\sigma = 0$. Then, $\dot{E(t)} > 0$.
\end{claim}

\begin{proof}
Differentiate \eqref{eq:earnings} and use \eqref{eq:lawh} to write
\begin{eqnarray}
\dot{E(t)} &=& R \dot{H(t)} - R \dot{I(t)H(t)} - P_{D} \dot{D(t)} \nonumber \\
           &=& R F \left( I(t) H(t), D(t) \right) - R \dot{I(t)H(t)} - P_{D} \dot{D(t)} \nonumber \\     
           &>& 0 
\end{eqnarray}

\noindent where the equality follows because the Marshallian demands for $I(t)H(t)$ and $D(t)$ are decreasing over time.
\end{proof}

\begin{claim} (Earnings over Time with no Depreciation) \label{claim:earndep}
Let $\sigma > 0$. Then, $\dot{E(t)} \lessgtr 0$. 
\end{claim}

\begin{proof}
Follow the same steps as in the proof of Claim~\ref{claim:earnnodep} and note that the term $R \sigma H(t)$ appears in the expression for $\dot{E(t)}$. This terms could be $\lessgtr R F \left( I(t) H(t), D(t) \right) - R \dot{I(t)H(t)} - P_{D} \dot{D(t)} $.
\end{proof}

\indent Claim \ref{claim:earnnodep} follows because the solution for human capital investment is interior $\forall \ t \in [0,T]$. Since there is no depreciation, individuals continuously add to their human capital stock and earn $R$ for each accumulated unity. Claim \ref{claim:earndep} follows because the interior solution for human capital investment may be driven down by a relatively high rate of depreciation.


\begin{proof}
Differentiate \eqref{eq:earnings} and use \eqref{eq:lawh} to write
\begin{eqnarray}
\dot{E(t)} &=& R \dot{H(t)} - R \dot{I(t)H(t)} - P_{D} \dot{D(t)} \nonumber \\
           &=& R F \left( I(t) H(t), D(t) \right) - R \dot{I(t)H(t)} - P_{D} \dot{D(t)} \nonumber \\     
           &>& 0 
\end{eqnarray}

\noindent where the equality follows because the Marshallian demands for $I(t)H(t)$ and $D(t)$ are decreasing over time.
\end{proof}

\subsection{Concavity} \label{section:egdyn}
\indent We now analyze the curvature of the earnings function for the case in which there is no depreciation.\footnote{A similar analysis follows when $\sigma>0$ for the cases in which either $\dot{E(t)}> 0$ or $\dot{E(t)}< 0$}. Without loss of generality we assume away $D(t)$, i.e. $F_{D(t)} = 0 $ so that the production function takes the single argument $I(t) H(t)$, and $R \equiv 1$. 

\begin{claim} (Concavity of the Earnings Function with no Depreciation) \label{claim:concearnnodep}
Assume $\eta \equiv \left( 1 - \frac{F'F'''}{{F''}^2} \right) < 0$. Then, the earnings function is strictly concave. 
\end{claim}

\begin{proof}
First note that $\dot{E(t)} > 0$ by Claim~\ref{claim:earnnodep}. Since $F_{D(t)} = 0 $ we can write the first order for investment becomes
\begin{equation}
g(t) F'(I(t) H(t)) = 1
\end{equation}
and we can differentiate it with respect to $t$ to get
\begin{eqnarray}
\dot{g(t)} F'(I(t) H(t)) + g(t) F''(I(t) H(t)) \dot{I(t) H(t)} &=& 0 \nonumber \\
&\Leftrightarrow& \nonumber \\
\dot{I(t) H(t)} &=& - \left( \frac{\dot{g(t)}}{g(t)} \right) \left[ \frac{F'}{F''}\right] \label{eq:itdot}.
\end{eqnarray}
\noindent Moreover, drop the argument $t$ to shorten notation, and note that
\begin{eqnarray}
\ddot{IH} = - \left[ \frac{\ddot{g}}{g} - \left( \frac{\dot{g}}{g} \right)^2 \right] \frac{F'}{F''} + \left( \frac{\dot{g}}{g} \right)^2 \left[ 1 - \frac{F'F'''}{{F''}^2} \right] \left[ \frac{F'}{F''} \right]  
\end{eqnarray}  
where we substitute in $\eqref{eq:itdot}$. Further, note that
\begin{eqnarray}
\dot{E} &=& F(IH) - \dot{IH} - \sigma H \nonumber \\
\ddot{E} &=& F'(IH) \dot{IH} - \ddot{IH} - \sigma \dot{H} \nonumber \\
&=& \frac{1}{g} \dot{IH} - \ddot{IH} - \sigma \dot{H}.
\end{eqnarray}

\noindent and from \eqref{eq:grossdep} obtain $\frac{\ddot{g}}{g} = r \frac{\dot{g}}{g}$. Thus,
\begin{eqnarray}
\ddot{E} &=& - \frac{\dot{g}}{g} \frac{F'}{F''} \left[ \frac{1}{g} - \frac{\dot{g}}{g} \left( 1 - \frac{F'F'''}{{F''}^2} \right) \right] + \left[ r \frac{\dot{g}}{g} - \left( \frac{\dot{g}}{g} \right)^2 \right] \frac{F'}{F''} \nonumber \\
&=& - \frac{\dot{g}}{g} \frac{F'}{F''} \left[ \frac{1}{g} - \frac{\dot{g}}{g} \left( 1 - \frac{F'F'''}{{F''}^2} \right) - \frac{gr - \dot{g}}{g} \right] \nonumber \\
&=& - \frac{\dot{g}}{g} \frac{F'}{F''} \left[ \frac{1}{g} - \frac{\dot{g}}{g} \left( 1 - \frac{F'F'''}{{F''}^2} \right) - \frac{1}{g} \right] \nonumber \\
&=& - \left( \frac{\dot{g}}{g} \right)^2 \frac{F'}{F''} \left( 1 - \frac{F'F'''}{{F''}^2} \right)
\end{eqnarray}

\noindent where the third equality uses \eqref{eq:grossdep}, i.e. $gr - \dot{g} = 1$. F is strictly concave and therefore $-\left( \frac{\dot{g}}{g} \right)^2 \frac{F'}{F''} > 0$. Since $\left( 1 - \frac{F'F'''}{{F''}^2} \right) < 0$ the claim follows.
\end{proof}

\indent A sufficient condition for $\dot{E(t)}$ to be concave is $\eta < 0$. This crucially depends on how concave $F'''$ is. Intuitively, given that there is no human capital depreciation, the production function for human capital investment needs to be concave enough to make the earnings function concave.

\begin{example} (Human Capital Production Functions and Earnings Concavity)
\begin{itemize}
\item Power Production Function 1 : consider the case of $F(x) = \frac{Ax^{\alpha}}{\alpha}$ for $ - \infty < \alpha < 1, A > 0$. Then, $\eta = \frac{1}{\alpha - 1} < 0$. Under this specification the earnings function is strictly concave with respect to time.
\item Power Production Function 2 : consider the case of $F(x) = a - b x ^ {- \alpha}$ for $ - 1 < \alpha < \infty, a,b,c > 0$. Then, $\eta = \frac{-1}{\alpha + 1} < 0$. Under this specification the earnings function is strictly concave with respect to time.
\item Power Production Function 3: consider the case of $F(x) = a - b \exp^{-cx}$ with $b,c > 0$. Then, $\eta = 0$. 
\item Quadratic Production Function: any quadratic production function has $F''' = 0$ and does not induce concavity of earnings with respect to time. 
\end{itemize}
\noindent Importantly, all this examples consider no depreciation of human capital, $\sigma = 0$. 
\end{example} 

\subsection{Specialization Period}
Specialization happens when the agent devotes his complete human capital to produce human capital stock, i.e. when $I(t) = 1$ for $t \in [ \underline{t}, \bar{t}]$. In order to analyze some of the properties of specialization periods we assume away $D(t)$ so that $F_{D(t)} = 0$ and rule out depreciation.\\
\indent Recall that we can interpret $g(t)$ as the return to investment in human capital investment and that we show above that in this case $\dot{g} < 0$. Then, there is at most one period of specialization at the beginning of the time horizon, if it happens. We denote this by $[0,t^*]$. \citet{ben1967production} calls this the schooling period and it happens under the conditions that follow.

\begin{condition} (Conditions for the Existence of a Period of Specialization in the Basic Ben-Porath Model with no Depreciation)
\begin{eqnarray}
F'(H(t))g(t) &>& R \nonumber \\
F'(H(t^*))g(t^*) &=& R \nonumber \\
I(t^*) &=& 1 \nonumber \\
H(t^*) &=& \int \limits _{0} ^{t^*} F (H(\tau)) d\tau + H_{0} \label{eq:humantstar}
\end{eqnarray} 
\noindent where $H(t^*)$ is the human capital stock accumulated up to time $t^*$, the sum over the period $[0,t^*]$ plus the initial stock.\\
\end{condition}

\indent Given that $R$ is fixed, any decrease in $g(t)$ lowers $t^*$ because it lowers the return to gross investment in human capital. For example, relatively high $r$ implies relatively low $t^*$ because the individual is relatively present oriented. Also, from \eqref{eq:humantstar}, note that a high value of $t^*$ implies a lower value for $t^*$ because it takes less time to obtain $H(t^*)$. If $\sigma > 0 $ the the same conditions characterize specialization. However, the there may be more than one specialization period because, under some scenarios, a high value of $\sigma$ may knock off capital such that various investment episodes are optimal. We differ that case for Section \ref{section:cycles}.
\begin{case} (No Depreciation and the Cobb-Douglas Production Function for Human Capital: Initial Human Capital and the Specialization Period) \label{case:ndcdexample}
In this case $\dot{H} = A \left(IH \right)^\alpha$ where $0 < \alpha < 1, A>0$. As argued above, specialization happens in the period $[0,t^*]$. Thus
\begin{eqnarray}
	\alpha A \left( H(0) \right) ^{\alpha - 1 }g(0) &>& R \nonumber \\
	&\Leftrightarrow& \nonumber \\
	H(0) &<& \left[ \frac{R}{g(0)\alpha A} \right]^{\frac{1}{\alpha-1}} \label{eq:h0forspe}.
\end{eqnarray}
\indent As the conditions in \eqref{eq:humantstar} establish, the time spent in specialization is a decreasing function of $H(0)$. In this example, actually, the initial human capital needs to be below certain threshold in order for the individual to specialize during one period. 
\end{case}

\begin{case} (No Depreciation and the Cobb-Douglas Production Function for Human Capital: Infinite Horizon, Initial Human Capital and the Specialization Period)
In the setting of Case~\ref{case:ndcdexample} and if the horizon of the problem is infinite: $H(0) < \left( \frac{\alpha A}{r} \right) ^{\frac{1}{1 - \alpha}}$ because $g(t) = \frac{R}{r}$.
\end{case}

\begin{case} (No Depreciation and the Cobb-Douglas Production Function for Human Capital: the Specialization Period)
In the period of specialization $I(t) = 1$. Then,
\begin{equation}
\dot{H} = A \left( H \right)^{\alpha} \label{eq:humandiff}.
\end{equation}
The general solution for \eqref{eq:humandiff} is
\begin{equation}
H(t) = \left[ (1 - \alpha)(At + K) \right]^{\frac{1}{1-\alpha}}
\end{equation}
for some constant $K$. Given an initial condition $H(0) = H_{0}$, $K = \frac{H_{0}^{1-\alpha}}{1-\alpha}$ and
\begin{equation}
H(t) = \left[ (1 - \alpha)At + H_{0}^{1-\alpha} \right]^{\frac{1}{1-\alpha}} \label{eq:hbeforetstar}.
\end{equation}
At the end of the specialization period, as established in \eqref{eq:humantstar}:
\begin{equation}
\alpha g(t^*) A \left( H(t^*) \right)^{\alpha - 1} = R.
\end{equation}
If $T \rightarrow \infty$, $g(t) = \frac{R}{r}$ and
\begin{equation}
t^* = - \frac{H_{0}^{1 - \alpha}}{A(1 - \alpha)} + \frac{\alpha}{1 - \alpha}\frac{1}{r}. \label{eq:tstar}
\end{equation}

\indent \eqref{eq:tstar} provides some intuitive results: (i) an individual with relatively high initial human capital specializes during a relatively shorter period: $\frac{\partial t^*}{\partial       H_{0}} < 0$; (ii) a relatively abler individual specializes during relatively longer period: $\frac{\partial t^*}{\partial A} > 0$; a relatively impatient individual specializes for a relatively shorter period: $\frac{\partial t^*}{\partial r} < 0$.
\end{case}

\begin{case} (No Depreciation and the Cobb-Douglas Production for Human Capital: Post-experience Earnings)
Let $\tau = t - t^*$ define the post-school work experience and write post-school earnings as follows:
\begin{equation}
E(\tau) = R \int \limits _{0} ^{\tau} \dot{H( l + t^*)}d l + R H(t^*) - RIH(\tau + t^*).
\end{equation}
Now, from \eqref{eq:humantstar} the following equality holds:
\begin{eqnarray}
\alpha g(t) A \left( IH(t) \right)^{\alpha - 1} &=& R \nonumber \\
&\Leftrightarrow& \nonumber \\
IH(t) &=& \left[ \frac{\alpha g(t) A}{R} \right]^{\frac{1}{1-\alpha}} \label{eq:itcobb}
\end{eqnarray}
Combining \eqref{eq:itcobb} and the law of motion for human capital:
\begin{equation}
\dot{H} = A \left[ \frac{\alpha g(t) A}{R} \right]^{\frac{\alpha}{1-\alpha}} \label{eq:hdot}.
\end{equation}
Then,
\begin{equation}
E(\tau) = R \int \limits _{0} ^{\tau} A \left[ \frac{\alpha g(l + t^*) A}{R} \right]^{\frac{\alpha}{1-\alpha}} dl + RH(t^*) - R \left[ \frac{\alpha g \left( \tau + t^* \right) A}{r} \right]^{\frac{1}{1-\alpha}} \label{equation:postearnings}
\end{equation}
and if $T \rightarrow \infty$
\begin{equation}
E(\tau) = RA \left[ \frac{\alpha A}{R} \right]^{\frac{\alpha}{1-\alpha}} \tau.
\end{equation}

\begin{center}
\begin{figure}[H]
\caption{Earnings and Experience, Cobb Douglas Technology and No Depreciation}
\centering
\includegraphics[width=3.5in, height=1.5in]{Figures/fig-earnings-experience.pdf}
\floatfoot{\begin{small}
Note:
\end{small}}
\end{figure}
\end{center}

\end{case}

\subsection{The Baseline Model Dynamics under the Cobb-Douglas Specification: a Summary}
This section summarizes the dynamics of the main variables in the baseline model when there is no depreciation, market goods are ruled out, and the production function for human capital investment is Cobb-Douglas. We assume that the horizon is infinite to simplify the algebra but it is important to remark that the qualitative properties of the results remain unchanged under finite horizon. To wrap up the section we show simulations that illustrate how the variables of interest behave under various parametrizations (in all of them we set $R = 1$). 

\subsubsection{Human Capital}
\begin{itemize}
\item At $t = 0$ an initial condition is given.
\item At $0 < t < t^*$ the system \eqref{eq:humantstar} provides the conditions that human capital satisfies and its expression is given by \eqref{eq:hbeforetstar}.
\item At $t = t^*$ \eqref{eq:hbeforetstar} is still a valid expression for human capital. To obtain the exact quantity it suffices to evaluate the expression for $t^*$, \eqref{eq:tstar}, into \eqref{eq:hbeforetstar}.
\item At $ t > t^* $ \eqref{eq:humantstar} and the expression for $\dot{H}$, \eqref{eq:hdot}, provide the expression for human capital.
\end{itemize}

Then,

\begin{eqnarray}
H(t) =
\begin{cases}
H_{0} & t = 0 \\
\left[ (1 - \alpha)At + H_{0}^{1-\alpha} \right]^{\frac{1}{1-\alpha}} , & 0 < t < t^* \\
\left[ \frac{\alpha A}{r} \right]^{\frac{1}{1 - \alpha}}, & t = t^* \\
\left[ \frac{\alpha A}{r} \right]^{\frac{ \alpha }{1 - \alpha}} \left( t - t^* \right) + \left[ \frac{\alpha A}{r} \right]^{\frac{1}{1 - \alpha}} , & t > t^*. \label{eq:humancapall}
\end{cases}
\end{eqnarray}

\subsubsection{Investment}
We focus on the case in which there is an specialization period, i.e. the case in which \eqref{eq:h0forspe} holds. The combination of \eqref{eq:itcobb} and \eqref{eq:humancapall} gives the following

\begin{eqnarray}
I(t) =
\begin{cases}
1, & t = 0 \\
1, & 0 < t < t^* \\
1, & t = t^* \\
\frac{\left[ \frac{\alpha A}{r} \right]^{\frac{1}{1 - \alpha}}}{\left[ \frac{\alpha A}{r} \right]^{\frac{ \alpha }{1 - \alpha}} \left( t - t^* \right) + \left[ \frac{\alpha A}{r} \right]^{\frac{1}{1 - \alpha}}}, & t > t^*. \label{eq:investall}
\end{cases}
\end{eqnarray}

\subsubsection{Earnings}
For the case of earnings we also on the case in which there is an specialization period, i.e. the case in which \eqref{eq:h0forspe} holds. Thus, \eqref{eq:earnings}, \eqref{eq:humancapall}, \eqref{eq:investall} define earnings as follows

\begin{eqnarray}
E(t) =
\begin{cases}
0, & t = 0 \\
0, & 0 < t < t^* \\
0, & t = t^* \\
RA \left[ \frac{\alpha A}{r} \right]^{\frac{\alpha}{1 - \alpha}} \left( t - t^* \right) , & t > t^*. \label{eq:earnsall}
\end{cases}
\end{eqnarray}

\begin{figure}[H]
     \begin{center}
    			\caption{Dynamics for $A = 3, r = .05, H_{0} = 1$ \\ $\alpha = .3$ (dotted); $\alpha = .4$ (dashed); $\alpha = .5$ (solid) }
        \subfigure[Human Capital Investment]{
            \includegraphics[width=2.2in, height=2.2in]{Figures/fig-hc-earn-series-02.pdf}
        }\\
        \subfigure[Human Capital Stock]{
            \includegraphics[width=2.2in, height=2.2in]{Figures/fig-hc-earn-series-01.pdf}
        }\\
        \subfigure[Earnings]{%
            \includegraphics[width=2.2in, height=2.2in]{Figures/fig-hc-earn-series-03.pdf}
        }
    \end{center}
\end{figure}

\begin{figure}[H]
     \begin{center}
    			\caption{Dynamics for $A = 3, \alpha = .5, H_{0} = 1$ \\ $r = .04$ (dotted); $r = .05$ (dashed); $r = .06$ (solid) }
        \subfigure[Human Capital Investment]{
            \includegraphics[width=2.2in, height=2.2in]{Figures/fig-hc-earn-series-05.pdf}
        }\\
        \subfigure[Human Capital Stock]{
            \includegraphics[width=2.2in, height=2.2in]{Figures/fig-hc-earn-series-04.pdf}
        }\\
        \subfigure[Earnings]{%
            \includegraphics[width=2.2in, height=2.2in]{Figures/fig-hc-earn-series-06.pdf}
        }
    \end{center}
\end{figure}

\begin{figure}[H]
     \begin{center}
   			\caption{Dynamics for $r = .03, \alpha = .5, H_{0} = 10$ \\ $A = .5$ (dotted); $A = 1.0$ (dashed); $A = 1.5$ (solid) }
        \subfigure[Human Capital Investment]{
            \includegraphics[width=2.2in, height=2.2in]{Figures/fig-hc-earn-series-08.pdf}
        }\\
        \subfigure[Human Capital Stock]{
            \includegraphics[width=2.2in, height=2.2in]{Figures/fig-hc-earn-series-07.pdf}
        }\\
        \subfigure[Earnings]{%
            \includegraphics[width=2.2in, height=2.2in]{Figures/fig-hc-earn-series-09.pdf}
        }
    \end{center}
\end{figure}

\begin{figure}[H]
     \begin{center}
    			\caption{Dynamics for $r = .025, \alpha = .5, A = .6$ \\ $H_{0} = 10$ (dotted); $H_{0} = 20$ (dashed); $H_{0} = 30$ (solid)}
        \subfigure[Human Capital Investment]{
            \includegraphics[width=2in, height=2in]{Figures/fig-hc-earn-series-11.pdf}
        }\\
        \subfigure[Human Capital Stock]{
            \includegraphics[width=2in, height=2in]{Figures/fig-hc-earn-series-10.pdf}
        }\\
        \subfigure[Earnings]{%
            \includegraphics[width=2in, height=2in]{Figures/fig-hc-earn-series-12.pdf}
        }
    \end{center}
\end{figure}

\subsection{Rates of Return under the Cobb-Douglas Specification}
We use this model to analyze returns both to schooling and post-schooling. In order to simplify the expressions we let $t \rightarrow T$ and we speculate that the analysis is not very sensitive to this assumption.

\subsubsection{Schooling}
We call schooling the period of specialization in which the individual devotes his complete human capital stock to produce human capital investment. To define the return to schooling consider two scenarios: (i) the individual does not invest either in schooling or in post-schooling. Each $\tau$ she earns $RH_{0}$; (ii) the individual invests during an specialization period and then does not invest after that. Each $\tau$ she earns $R \left[ \frac{\alpha A}{r} \right]^{\frac{1}{1-\alpha}}$. Then, we can define the (internal) rate of return of schooling as follows.

\begin{definition} (``Internal'' Rate of Return to Schooling)
$\varphi$ is the (internal) rate of return to schooling and solves the equation
\begin{eqnarray}
\int _{t^*} ^{\infty} \exp^{- \varphi t} R \left[ \frac{\alpha A}{r} \right]^{\frac{1}{1-\alpha}} dt &=& \int \limits _{0} ^{\infty} \exp^{- \varphi t} R H_{0} dt \nonumber \\
&\Rightarrow& \nonumber \\
\varphi &=& \frac{\ln \left[ \frac{\alpha A}{r} \right]^{\frac{1}{1 - \alpha}} - \ln H_{0}}{\frac{1}{r} - \frac{1}{2} \frac{H_{0}^{\frac{1}{2}}}{A}}.
\end{eqnarray}
\end{definition}

\subsubsection{Post-schooling}
Let $E(\tau)^{NPS}$ and $E(\tau)^{PS}$ denote earnings without and with post-schooling investment, respectively. By \eqref{eq:earnsall} we can write
\begin{eqnarray}
E(\tau)^{NPS} &=& R H(t^*) \\ \nonumber
&=& R \left( \frac{\alpha A}{r} \right)^{\frac{1}{1 - \alpha}} \\ \nonumber
E(\tau)^{PS} &=& \left[ \frac{\alpha A}{r} \right]^{\frac{\alpha}{1-\alpha}} \tau
\end{eqnarray}

\noindent so that the increment in earnings due to post-schooling at $\tau$ is
\begin{eqnarray}
\Delta^{E(\tau)} \equiv E(\tau)^{PS} - E(\tau)^{NPS}.
\end{eqnarray}

\indent Actually, note that $E(\tau)^{PS} = IH(\tau)$ (see \eqref{eq:itcobb}) so that we can interpret $\Delta^{E(\tau)}$ as ``returns less costs'' from post-schooling. Then, we define the (internal) rate of return to post-schooling as follows.

\begin{definition} (``Internal'' Rate of Return to Post-schooling)
$\phi$ is the (internal) rate of return to schooling and solves the equation
\begin{equation}
\int \limits _{0} ^{\infty} \exp^{- \phi \tau} \left[ \left[ \frac{\alpha A}{r} \right]^{\frac{\alpha}{1-\alpha}} \tau - R \left( \frac{\alpha A}{r} \right)^{\frac{1}{1 - \alpha}} \right] d \tau = 0 \label{eq:postreturn}
\end{equation}
\noindent Using the Laplace transform, \eqref{eq:postreturn} implies
\begin{eqnarray}
\frac{1}{\phi^2} RA \left[ \frac{\alpha A}{r} \right]^{\frac{\alpha}{1-\alpha}} + \frac{1}{\phi} A \left[ \frac{\alpha A}{r} \right]^{\frac{1}{1-\alpha}} &=& 0 \nonumber \\
&\Rightarrow& \nonumber \\
\phi &=& \frac{r}{\alpha}.
\end{eqnarray}
\end{definition}

\indent The (internal) rate of return to post-schooling investment is a decreasing function of $\alpha$. Individuals who are more productive require a smaller return in order to invest in the post-school period. Likewise, relatively patient individuals (relatively low $r$) require a smaller $\phi$ to invest in the post-schooling period.

\subsection{Earnings Growth and Patience in Finite Horizon}
In this section we want to ask, in the same framework, how earnings growth depend on what defines relative patience in this model, the discount rate $r$. To do that, we investigate $\frac{\partial \dot{E(\tau)}}{\partial r}$.

\begin{claim} \label{claim:moreconcwithighr}
Assume that $1 - \frac{F'(\cdot) F'''(\cdot)}{{F''}^2} < 0$ (recall from Claim \ref{claim:concearnnodep} that this is a sufficient condition for $\dot{E(t)}$ in the current context). Then, $\frac{\partial \dot{E(\tau)}}{\partial r} < 0$.
\end{claim} 

\begin{proof}
Without loss of generality, assume that R =1 and note that\\
\begin{equation}
\frac{\partial \dot{E(\tau)} }{\partial r} = F'(\cdot) \frac{\partial IH}{\partial r} - \frac{\partial}{\partial r} \dot{IH}. \label{eq:earnpartialr}
\end{equation}

\noindent From \eqref{eq:newfocs} we know that the first order condition of the agent's problem is
\begin{equation}
g(t) F'(\cdot) = 1
\end{equation}

\noindent which by the implicit function theorem yields
\begin{eqnarray}
\frac{\partial IH}{\partial r} &=& \frac{\frac{\partial g(t)}{\partial r} F'(\cdot)}{2 g(t) F''(\cdot)} \nonumber \\ 
&<& 0
\end{eqnarray}

\noindent where the inequality follows from strict concavity of $F(\cdot)$ and  $\ g(t) > 0, \frac{\partial g(t)}{ \partial r} <0$ (see \eqref{eq:partialgr}). Thus, the first term in \eqref{eq:earnpartialr} is negative. If we show that the second term is negative then we can sign \eqref{eq:earnpartialr} and provide a meaning for this results. In order to do that we need $\frac{\partial \dot{IH}}{\partial r} > 0$. From \eqref{eq:itdot} note that 
\begin{equation}
\frac{\partial \dot{IH}}{\partial r} = -\frac{\dot{g}}{g} \left[ 1 - \frac{F'(\cdot)F'''(\cdot)}{{F''(\cdot)}^2}\right] \frac{\partial IH}{\partial r} + \frac{F'(\cdot)}{F''(\cdot)}\frac{\partial}{\partial r} \left[ - \frac{\dot{g}}{g} \right] \label{eq:ihdotr}
\end{equation}

\indent We know that $1 - \frac{F'(\cdot) F'''(\cdot)}{{F''}^2} < 0$ and $\dot{g}, \frac{\partial IH}{\partial r} < 0$  the first term in \eqref{eq:ihdotr} is positive. To sign the second term note that $\dot{g} = rg - 1, - \frac{\dot{g}}{g} = \frac{1}{g} - r$. Then,
\begin{equation}
\frac{\partial}{\partial r} \left[ - \frac{\dot{g}}{g} \right] = - \frac{1}{g^2} \frac{\partial g}{\partial r} - 1. \label{eq:dotgg}
\end{equation}

\indent To sign \eqref{eq:dotgg} note that
\begin{eqnarray}
\frac{\partial g}{\partial r} &=& \frac{\exp^{r(t-T)} \left( 1 - r(t - T) \right) - 1 }{r^2} \nonumber \\
&<& 0 \label{eq:partialgr}
\end{eqnarray}

\noindent and

\begin{eqnarray}
- \frac{\partial g }{g^2 \partial r} - 1 &=& \frac{1}{r^2g^2} \exp^{r(t-T)} \left( 1 + r(t - T) - \exp^{r(t-T)} \right). \nonumber \\
&<& 0
\end{eqnarray}

\noindent which implies that $\frac{\partial \dot{E}}{\partial r} < 0$. 
\end{proof}

\indent The graphical representation of Claim~\ref{claim:moreconcwithighr} is in Figure~\ref{fig:earnprofr}. It implies that the earnings function is relatively ``less concave'' for relatively impatient individuals (relatively high $r$). This is a consequence of their investment decisions: they spent less time in the schooling period and accumulate less human capital.

\begin{center}
\begin{figure}[H]
\caption{Earnings Profiles in Finite Horizon for Different Values of $r$ } \label{fig:earnprofr}
\centering
\includegraphics[width=3.5in, height=1.5in]{Figures/fig-earnings-growth.pdf}
\floatfoot{\begin{small}
\end{small}}
\end{figure}
\end{center} 

\section{The Haley-Rosen Specification: Finite Horizon and the Autoregression Form}

\citet{haley1976estimation} and \citet{rosen1976theory} consider a model in which  Assumption~\ref{assumption:bbpmodel} and Assumption~\ref{assumption:scpf} hold. They further assume that $\dot{H} = A \left( IH \right)^{\frac{1}{2}}$. This is, they assume that the production function for human capital investment is Cobb-Douglas with an specific value for $\alpha$ and that there is no depreciation.

\begin{definition} (Law of Motion for Human Capital Stock in the Haley-Rosen Specification)
\begin{equation}
\dot{H} = A \left( IH \right)^{\frac{1}{2}} \label{eq:halroslaw}.
\end{equation}
\end{definition}

\indent In Section \ref{section:baseline}, some of the results rely on infinite horizon to derive a set of closed form solutions to the individual's problem. This specification allows for tractability of the finite horizon case. In particular, we focus on the dynamics of post-schooling earnings because one of the implications of the infinite horizon case in Section~\ref{section:baseline} is the linear relation between earnings and experience an may be a caveat.\footnote{Estimates of the so-called Mincer equation usually establish a non-linear relation between earnings an experience \citep[see][]{heckman2006earnings}}\\

\begin{claim} \label{claim:concavehalros} (Concavity of Earnings in the Haley-Rosen Specification)
Consider the Haley-Rosen Specification of the Basic Ben-Porath Model explained in Section \ref{section:baseline}, i.e. assume that \eqref{eq:halroslaw} is the law of motion for human capital. Then, the earnings function is strictly concave in experience when the time horizon is finite.
\end{claim}

\begin{proof}
\indent From \eqref{equation:postearnings} we can write
\begin{eqnarray}
E(\tau) &=& RH(t^*) + R \int \limits _{0} ^{\tau} A \left[\frac{1}{2} \frac{g(t^* + l)A}{R} \right]dl - R \left[\frac{1}{2} \frac{g(t^* + \tau)A}{R} \right]^{2} \nonumber \\
&\Rightarrow& \nonumber \\
\dot{E(\tau)} &=& \frac{g (t^* + \tau)A^2}{2R}\left( 2R -rg (t^* + \tau) \right) \nonumber \\
&\Rightarrow& \nonumber \\
\ddot{E(\tau)} &=& -\frac{A^2}{R}\dot{g(t^* + \tau)}^2 \label{eq:earnalpha2}
\end{eqnarray}
where the second and third equalities use \eqref{eq:grossdep}. Combining \eqref{eq:grossdep} and \eqref{eq:earnalpha2} we obtain a second order ODE with constant coefficients:
\begin{equation}
\ddot{E(\tau)} = 2r \dot{E(\tau)} - A^2 R \label{eq:earndifferential}
\end{equation} 

\noindent where the natural initial and terminal conditions that we impose are $E(0)$ and $\ddot{E(T)} = 0$ and then we guess and verify that $c_{2} = \frac{A^2 R}{2r}\exp^{2rT}$ in the following general solution to \eqref{eq:earndifferential}
\begin{equation}
E(\tau) = c_{0} + c_{1}\exp^{-2r \tau} + c_{2} \tau
\end{equation}

\noindent so that $c_{1} + c_{0} = 0, 2rc_1 \exp^{2rT}  + c_{2} = 0$ and, therefore,
\begin{equation}
E(\tau) = \frac{A^2 R}{4r^2} \exp^{-2r T} \left( 1 - \exp^{2r \tau} \right) + \frac{A^2 R}{2r} \tau \label{eq:earningspostalpha12}
\end{equation}

\noindent which is strictly concave in $\tau$.
\end{proof}


\indent The graphical representation of Claim~\ref{claim:concavehalros} is in Figure~\ref{fig:halros}. 
\begin{center}
\begin{figure}[H]
\caption{Post-school Earnings in the Haley-Rosen Specification} \label{fig:halros}
\centering
\includegraphics[width=3.5in, height=1.5in]{Figures/fig-finite-horiz.pdf}
\floatfoot{\begin{small}
Note:
\end{small}}
\end{figure}
\end{center}

\noindent As $t \rightarrow T$ the earnings function becomes less concave on experience. This follows from the trade-off that the individual faces when investing in the two different time horizons. When the individual faces an infinite horizon it is optimal for him to keep investing in human capital at a constant rate, even after the specialization period, as times goes by. In finite horizon, however, the incentives to invest decrease over time, and therefore, earnings increase at a decreasing rate. 

\subsection{Evidence}
\citet{brown1976model} estimates \eqref{eq:earningspostalpha12}, which enables to identify $r$ and $A^2 R$. His estimates, however, are imprecise and show that $r \rightarrow 0$. Then, he estimates the model for the infinite horizon case. He claims this to be a good approximation because he has a sample of young individuals. However, this disables him to estimate $r$.

\subsection{The Autoregression}
The Haley-Rosen specification enables straightforward analysis of the earnings dynamics. From \eqref{eq:earningspostalpha12} it is possible to write
\begin{equation}
E(\tau + 1) - E(\tau) =  \frac{A^2 R}{2r}+ \frac{A^2 R}{4r^2} \exp^{-2rT} \left( \exp^{2r(\tau - 1)} \exp^{r \tau} \right) 
\end{equation}

\noindent which implies that

\begin{equation}
z(\tau + 1) = \exp^{2r} z(\tau) + \frac{A^2 R}{2r} \left( 1 - \exp^{2r} \right) \label{eq:growthalpha12}
\end{equation}

\noindent where $z(\tau) \equiv  E(\tau + 1) - E(\tau)$. The graphical representation of \eqref{eq:growthalpha12} in Figure~\ref{fig:halrosgrowth} may mislead to the conclusion that earnings growth follow an explosive dynamic. Claim~\ref{claim:halrosdyn} clarifies this. 

\begin{center}
\begin{figure}[H]
\caption{Earnings Growth in the Haley-Rosen Representation} \label{fig:halrosgrowth}
\centering
\includegraphics[width=3in, height=3in]{Figures/fig-explode-converge.pdf}
\floatfoot{\begin{small}
\end{small}}
\end{figure}
\end{center}

\begin{claim} \label{claim:halrosdyn}
Earnings growth, $z(\tau)$, converges to a constant.
\end{claim}

\begin{proof}
Informally, note that
\begin{eqnarray}
\frac{\partial \left[ E(\tau) - E(\tau - 1) \right] }{\partial \tau} &=& \frac{A^2 R}{2r} \exp{2r(\tau - T)} \left[ \exp^{-2r} - 1 \right] \nonumber \\
&<& 0.
\end{eqnarray}

\noindent Formally, note that $z(0) = E(0) \equiv z_{0}$ and solve \eqref{eq:growthalpha12} to get
\begin{equation}
z(\tau) = \exp^{2rT} z_{0} + \frac{A^2 R}{2r} \left( 1 - \exp{2r} \right) \sum \limits _{j=0} ^{T-1} \exp^{2rj}
\end{equation}

\noindent so that the earnings growth converges to the constant $\exp^{2rT} z_{0}$.
\end{proof}

\subsection{From the Haley-Rosen Specification to the Mincer Equation}
The earnings function in the Haley-Rosen specification actually leads to the Mincer equation. To see that take take logs of \eqref{eq:earningspostalpha12} and obtain
\begin{equation}
\ln E(\tau) = \ln \left( \frac{A^2 R}{ 2r} \right) + \ln \tau + \ln \left[ 1 + \frac{\exp^{-2rT} - \exp^{2r(\tau - T)} }{2r \tau} \right]. \label{eq:logs}
\end{equation}

\indent We can approximate around $\tau_{0}$ the second and third terms in \eqref{eq:logs} to obtain
\begin{eqnarray}
\ln(\tau) &\approx& \ln (\tau_{0}) + \frac{1}{\tau_{0}} \left( \tau - \tau_{0} \right) - \frac{1}{\tau_{0}^2} \frac{\left( \tau - \tau_{0} \right)^2}{2!} \nonumber \\
\ln \left[ 1 + \frac{\exp^{-2rT} - \exp^{2r(\tau - T)} }{2r \tau} \right] &\approx& \xi_{0} + \xi_{1} \left( \tau - \tau_{0} \right) + \xi_{2} \frac{\left( \tau - \tau_{0} \right)^2}{2!}
\end{eqnarray}

\noindent for the adequate $\xi_{0}, \xi_{1}, xi_{2}$. Thus,
\begin{equation}
\ln(\tau) + \ln \left[ 1 + \frac{\exp^{-2rT} - \exp^{2r(\tau - T)} }{2r \tau} \right] \approx \alpha_{0} + \alpha_{1}\left( \tau - \tau_{0} \right) + \alpha_{2} \left( \tau - \tau_{0} \right)^2
\end{equation}

\noindent with $\alpha_{0} \equiv \ln(\tau_{0}) + \xi_{0}, \alpha_{1} \equiv \frac{1}{\tau_{0}} + \xi_{1}, \alpha_{2} \equiv \frac{-\frac{1}{\tau_{0}^2} + \xi_{2}}{2}$, which leads to the so-called Mincer equation \citep[see][]{mincer1974schooling}:
\begin{equation}
\ln E(\tau) = k_{0} + k_{1} \tau k_{2} \tau^2 \label{eq:mincer}
\end{equation}

\noindent where $k_{0} = \alpha_{0} - \tau_{0} \alpha_{1} + \alpha_{2} \tau_{0}^2, k_{2} = \alpha_{2}$. This provides a baseline to compare ``Ben-Porath'' with ``Mincer'' coefficients. Table \ref{table:bpmincer} provides different combinations of the parameters $r, \tau_{0}, T$ that lead to different values of $k_{1}, k_{2}$ that are close to the estimates that \citet{mincer1974schooling} obtains.

\begin{center}
\begin{table}[H]
\begin{threeparttable}
\fontsize{9}{12pt}\selectfont
\caption{The Ben-Porath and the Mincer Coefficients} \label{table:bpmincer}
\centering
\begin{tabular}{ccc|cc}
\multicolumn{3}{c|}{Parameters} & \multicolumn{2}{c}{Ben Porath
Coefficients} \\ \hline\hline $r$ & $\tau _{0}$ & $T$ & $k_{1}$ &
$k_{2}$ \\ \hline $0.0225$ & $29.54$ & $41.43$ & $0.081$ & $-0.0010$
\\ \hline $0.05$ & $25$ & $60$ & $0.0808$ & $-0.0008$ \\ \hline
$0.05$ & $20$ & $65$ & $0.1002$ & $-0.0013$ \\ \hline $0.0675$ &
$24.70$ & $74.77$ & $0.081$ & $-0.0008$ \\ \hline\hline
\multicolumn{3}{c}{Mincer Coefficients} & $0.081$ & $-0.0012$ \\
\hline
\end{tabular}
\begin{tablenotes}
\small
\item Note: the Mincer model or Mincer equation is $\ln ( \text{E} ) =k_{0}+k_{1}\tau +k_{2}\tau^{2}$, where $\tau$ is experience.  
\end{tablenotes}
\end{threeparttable}
\end{table}
\end{center}

\indent Now, if $rT \approx 0$ then $\exp^{-rT} \approx 1$ and \eqref{eq:logs} becomes 
\begin{equation}
\ln E(\tau) \approx \ln \left( \frac{A^2 R}{2r} \right) + \ln \tau + \ln \left[ 1 + \frac{1 - \exp^{2r \tau}}{2 r \tau} \right] \label{eq:intercept}
\end{equation}

\noindent which leads to various observations. The Haley-Rosen specification of the Ben-Porath model implies no economic content for the Mincerian rate of return on post-school investment. Put differently, an extension of \eqref{eq:mincer} which includes post-school investment does not have a structural counterpart. Actually, this model implies that the entire economic content is in the intercept (see \eqref{eq:intercept}. Actually, \eqref{eq:intercept} implies that, \textit{caeteris paribus}, schooling has no effect on earnings. \citet{mincer1974schooling} finds that the contrary. However, we claim that his finding does not necessarily argues against the Ben-Porath model. It could simply be the case that \citet{mincer1974schooling} does not include ability measures in his estimations, which appear in \eqref{eq:intercept}, and therefore finds a positive coefficient on schooling. 

 
\section{Generalized Ben-Porath Model} \label{section:generalized}
We know generalize the model in Section \ref{section:baseline} to a more general production function of human capital investment. We focus our analysis on specialization because the analysis of other conditions is very similar to that of Section \ref{section:baseline}

\begin{definition} (Law of Motion for Human Capital Stock in the Generalized Ben-Porath Model)
\begin{equation}
\dot{H} = A I^{\alpha} H^{\beta} - \sigma H \label{eq:lawhgen}.
\end{equation}
\end{definition}

\indent The model in Section \ref{section:baseline} is a particular case of this general formulation when $\alpha = \beta$. To simplify the analysis of the implications of this model we assume that there is neither discounting nor depreciation, i.e. $r = \sigma = 0$. To ease notation we neglect the argument $t$ when possible. We analyze this model in finite horizon.\\
\indent The Hamiltonian of the problem is
\begin{equation}
\mathcal{H} = RH(t) \left(1 - I(t) \right) + \mu \left( A I^{\alpha} H^{\beta} \right)
\end{equation} 

\noindent where $\mu(t)$ defines the shadow price of human capital. The following condition describe optimality.

\begin{condition} (Optimality Conditions for the Life-Cycle Individual's Problem in the Generalized Ben-Porath Model) \label{condition:optgen}
\begin{eqnarray}
\frac{\partial \mathcal{H} (\cdot)}{\partial I(t)} = 0 &\Leftrightarrow& \mu A \alpha I^{\alpha - 1} H^{\beta} \geq RG \label{eq:focinvestmentgen} \\
\frac{\partial \mathcal{H} (\cdot)}{\partial H(t)} = - \dot{\mu(t)} &\Leftrightarrow& -R(1 - I) - \beta \mu A I^{\alpha} H^{\beta -1} \label{eq:focstockgen} \\ 
\frac{\partial \mathcal{H} (\cdot)}{\partial \mu(t)} = \dot{H} &\Leftrightarrow& \dot{H(t)} = F \left( I(t) H(t), D(t) \right) - \sigma H(t) \label{eq:focmotiongen} \\
\text{Transversality} &:& \lim_{t \rightarrow T} \mu(t) H(t) = 0 \label{eq:foctransversalitygen}
\end{eqnarray}
\end{condition}

\indent Condition \ref{condition:optgen} are equivalent to the Mangasarian sufficient conditions for a global optimum if $\beta \leq 1$ \citep[see][]{mangasarian1966sufficient}.

\subsection{Specialization}
If \eqref{eq:focinvestmentgen} holds with strict inequality the individual specializes, i.e. $I=1$. Thus, Condition~\ref{condition:spe} guarantees specialization.

\begin{condition} (Conditions for Specialization in the Generalized Ben-Porath Model) \label{condition:spe}
\begin{eqnarray}
\text{Conditions for Specialization :}
\begin{cases}
H > \left[ \frac{R}{\alpha A \mu} \right]^{\frac{1}{\beta - 1}}, & \beta > 1 \\
1 > \left[ \frac{R}{\alpha A \mu} \right]^{\frac{1}{\beta - 1}}, & \beta = 1 \\
H < \left[ \frac{R}{\alpha A \mu} \right]^{\frac{1}{\beta - 1}}, & \beta < 1. \\
\end{cases}
\end{eqnarray}
\end{condition}

\indent During the period(s) of specialization \eqref{eq:focstockgen}, \eqref{eq:focmotiongen} become 
\begin{eqnarray}
\dot{\mu} &=& - \beta \mu A H^{\beta - 1} \label{eq:focstockgenspe} \\
\dot{H}  &=& A H^{\beta} \label{eq:focmotiongenspe}\\
\end{eqnarray}

\noindent and we can solve for the dynamics of human capital stock in this region
\begin{eqnarray}
H(t) =
\begin{cases}
c_{0} \exp^{At}, & \beta = 1 \\ 
\left( At + c_{1} \right)^{\frac{1}{1 - \beta}} (1 - \beta)^{\frac{1}{1 - \beta}}, & \beta \neq 1.

\end{cases}
\end{eqnarray}

\noindent The initial condition for the human capital stock leads to $c_{0} = \frac{H_{0}}{\exp^{1}} $ and $c_{1} = \frac{H_{0}^{1 - \beta}}{1-\beta}$ which implies that
\begin{eqnarray}
H(t)
\begin{cases}
H_{0} \exp^{At - 1}, & \beta = 1 \\
\left( At + \frac{H_{0}^{\frac{1}{1-\beta}}}{1-\beta} \right)^{\frac{1}{1 - \beta}} \left( 1 - \beta \right)^{\frac{1}{1-\beta}}, & \beta \neq 1. \label{eq:humanspe}
\end{cases}
\end{eqnarray}

\indent Also, we can solve \eqref{eq:focstockgenspe} and find that
\begin{eqnarray}
\mu(t) =
\begin{cases}
k_{0} \exp^{-At}, & \beta = 1 \\
\frac{k_{1}}{(At + c_{1})^{\frac{\beta}{1-\beta}}}, & \beta \neq 1 \label{eq:muspe}
\end{cases}
\end{eqnarray}

\noindent for which there is an exact solution given an initial condition $\mu(0) = \mu_{0}$. This is, we can find $k_{0}, k_{1}$ in \eqref{eq:muspe} provided $\mu_{0} > 0$ (it is a price). In particular, note that $k_{0} = \mu_{0} > 0$ and $k_{1} = \mu_{0} c_{1}^{\frac{\beta}{1-\beta}} > 0$ for $0<\beta<1$.\\
\indent Let $t^*$ denote the time when specialization ends. It must be true that, then, \eqref{eq:focinvestmentgen} holds with strict equality
\begin{equation}
\mu(t^*) A \alpha H(t^*)^{\beta} = RH(t^*)
\end{equation}

\noindent which implies that 
\begin{equation}
t^* = \frac{1}{A} \left( \ln \left[ \frac{A\alpha}{R} + \ln k_{0} \right] \right)
\end{equation}

\noindent for $\beta = 1$. For $\beta \neq 1$, $t^*$ solves
\begin{equation}
\frac{k_{1}}{ \left( At^* + c_{0} \right)^{\frac{\beta}{\beta-1}}} \frac{A \alpha}{R} = \left[ At^* \left( 1 - \beta \right)^{\frac{1}{1 - \beta}} + H_{0}^{1 - \beta} \left( 1 - \beta \right)^{\frac{\beta}{1 - \beta}} \right]^{1 - \beta}.
\end{equation}

\indent To wrap up the discussion we ask if the period of specialization is unique for some particular cases.

\begin{claim} (Uniqueness of the Specialization Period)
The period of specialization is unique when either $\beta = 1$ or $\beta \in [0,1]$. 
\end{claim}

\begin{proof}
In both cases \eqref{eq:muspe} implies that $\dot{\mu(t)} < 0$. Importantly, $\mu(t)$ is the shadow price or value of human capital. Thus, $\dot{I(t)} < 0 $ and, if it exists, the period of specialization is unique.
\end{proof}

 
\section{The Basic Shenshinski Specification}
The Basic Shenshinski specification is a particular case of the Generalized Ben Porath model in Section \ref{section:generalized} in which $\alpha = \beta = 1$. Then, $\dot{H(t)} = AI(t)H(t) - \sigma H(t)$. Proceeding in Section \ref{section:baseline} and Section \ref{section:generalized} we can write down the current value Hamiltonian and obtain the following conditions for optimality
\begin{eqnarray}
\frac{\partial \mathcal{H} (\cdot)}{\partial I(t)} = 0 &\Leftrightarrow& \mu(t) \exp{rt} \geq \frac{R}{A} \label{eq:focinvestmentshenbasic} \\
\frac{\partial \mathcal{H} (\cdot)}{\partial H(t)} = - \dot{\mu(t)} &\Leftrightarrow& - \exp^{rt} R(1 - I(t)) - \mu(t) \left( A I(t) - \sigma \right) = \dot{\mu(t)} \label{eq:focstockshenbasic} \\ 
\frac{\partial \mathcal{H} (\cdot)}{\partial \mu(t)} = \dot{H(t)} &\Leftrightarrow& \dot{H(t)} =  A I(t) H(t)- \sigma H(t) \label{eq:focmotionshenbasic} \\
\text{Transversality} &:& \lim_{t \rightarrow T} \mu(t) H(t) = 0 \label{eq:foctransversalityshenbasic}
\end{eqnarray}

\indent We want to analyze the investment dynamics. To do that, define $g(t) = \mu(t) \exp^{rt}$. Use \eqref{eq:focstockshenbasic} and \eqref{eq:foctransversalityshenbasic} to obtain
\begin{eqnarray}
\dot{g} &=& -R + (R - Ag)I + (\sigma + r)g \label{eq:gbasicshen} \\ 
g(T) &=& 0.
\end{eqnarray}

\indent In the specialization period $I(t) = 1$ and from \eqref{eq:gbasicshen} and \eqref{eq:focinvestmentshenbasic} we have $g(t) = g_{0                                                                                                                                                                                                                                                                                                                                                                                                                                                                                                                                                                                                                                                                                                                                                                                                                                                                                                                                                                                                                                                                                                                                                                                                                                                                                                               } \exp^{(\sigma + r - A)t}$. If $\sigma + r > A$ the solution violates \eqref{eq:foctransversalityshenbasic}: $\dot{g(t)}>0$ and $I(t) = 1 \ \forall t \in [0,T]$. If $\sigma + r < A$, $\dot{g(t)}<0$. Actually, by \eqref{eq:focinvestmentshenbasic}, $g(t)$ decreases to $\frac{R}{A}$ and $I(t) = 0$ onwards. This happens because $g(t)$ represents returns to gross investment in human capital net accounting for time and depreciation and $I(t)$ is bounded below by zero. Thus, if $g(t)$ reaches a constant phase $I(t)$ remains constant at $0$. Now, when $I(t) = 0$ we can use $g(T) = 0$ and write
\begin{eqnarray}
\dot{(g(t))} &=& (\sigma + r)g(t) - R \nonumber \\
&\Rightarrow& \nonumber \\ 
g(t) &=& \frac{R}{\sigma + r} \left[1 - \exp^{(\sigma + r)(t-T)} \right] \label{eq:gshenbasei0}.
\end{eqnarray}

\noindent for which $\dot{g(t)}<0$ as well. Then, once $I(t)$ reaches zero it never goes back again to a positive value. This formulation has a Bang-Bang equilibrium.

\begin{center}
\begin{figure}[H]
\caption{Bang-Bang Equilibrium in the Basic Shenshinski Specification}
\centering
\includegraphics[width=3.5in, height=3.5in]{Figures/fig-shesh-rule.pdf}
\floatfoot{\begin{small}
Note:
\end{small}}
\end{figure}
\end{center}

\indent It follows that the schooling period is unique and at the beginning of the investment cycle.\footnote{Mistake in slide 7 when solving for $t^*$.} We can actually solve for $t^*$ the length of the schooling period using the fact that $g(t^*) = \frac{R}{A} $ by \eqref{eq:focinvestmentshenbasic} and $g(t^*) = \frac{R}{\sigma + r} \left[1 - \exp^{(\sigma + r)(t^*-T)} \right]$ by \eqref{eq:gshenbasei0}. This implies that (i) longer investment horizons imply more schooling, $\frac{\partial t^*}{\partial T} > 0$; (ii) greater depreciation implies less schooling, $\frac{\partial t^*}{\partial \sigma} < 0$; (iii) higher relative impatience implies less schooling, $\frac{\partial t^*}{\partial r} > 0$; (iv) higher productivity implies more schooling; (v) initial human capital does not affect schooling, $\frac{\partial t^*}{\partial H_{0}} = 0$. This is because
\begin{eqnarray}
\frac{R}{A} &=& \frac{R}{\sigma + r} \left[1 - \exp^{(\sigma + r)(t^*-T)} \right] \nonumber \\
&\Leftrightarrow& \nonumber \\ 
t^* &=& \frac{1}{\sigma + r} \ln \frac{A - (\sigma + r) }{A} + T.
\end{eqnarray}

\subsection{From the Basic Shenshinski Specification to the Mincer Equation}
From \eqref{eq:focmotiongenspe} we know that in the period $[0,t^*]$\footnote{Mistake in slide 8. Why add $\varphi$? Makes no sense.}
\begin{eqnarray}
\dot{H(t)} = (A - \sigma)H(t) \nonumber \\
&\Rightarrow& \nonumber \\ 
H(t) = H_{0} \exp^{A - \sigma}.
\end{eqnarray}

\noindent At $t^*$, actually, $I(t) = 0$ so earnings are $Y(t) = R H(t^*)$.\footnote{Miss-defined in slide 8} Then,
\begin{equation}
\ln Y(t^*) = \ln(RH_{0}) + (A - \sigma)t*.
\end{equation} 

\noindent According to this model, the returns to schooling are given by the productivity of the human capital investment production function less the human capital depreciation.  

\section{The Modified Shenshinski Specification}
Consider, now, the law of motion for human capital $\dot{H} = AI - \sigma H$.\footnote{Typo in slide 21 in the definition of the law of motion and in the definition of the optimality conditions}. The Hamiltonian of the problem is
\begin{equation}
\mathcal{H}(\cdot) = \exp^{-rt} R (1 - I)H + \mu (AI - \sigma).
\end{equation}

\noindent The corresponding optimality conditions are
\begin{eqnarray}
\frac{\partial \mathcal{H} (\cdot)}{\partial I(t)} = 0 &\Leftrightarrow& \mu \exp^{rt} \geq \frac{RH}{A} \label{eq:focinvestmentmodsehn} \\
\frac{\partial \mathcal{H} (\cdot)}{\partial H(t)} = - \dot{\mu(t)} &\Leftrightarrow& \dot{\mu} = \mu \sigma - \exp^{-rt} R (1 - I) \label{eq:focstockmodshen} \\ 
\frac{\partial \mathcal{H} (\cdot)}{\partial \mu(t)} = \dot{H(t)} &\Leftrightarrow& \dot{H(t)} = AI - \sigma H \label{eq:focmotionmodshen} \\
\text{Transversality} &:& \lim_{t \rightarrow T} \mu(t) H(t) = 0. \label{eq:foctransversalitymodshen}
\end{eqnarray}

\indent Define $g(t) = \mu(t) \exp^{rt}$ and use \eqref{eq:focstockmodshen} to obtain
\begin{equation}
\dot{g} = g(\sigma + r) - R(1 - I). \label{eq:gmodshen}
\end{equation}

\subsection{No Depreciation: a Schooling Model}
Let $\sigma = 0 $. Then $\dot{g} = -R(1 - I) + rg$ ans $\dot{H} = A$ and we obtain the solution for the human capital trajectory when $I = 1$
\begin{equation}
H(t) = At + H_{0} \label{eq:humancapshenmod}.
\end{equation} 

\noindent At $t = 0$, $I = 1$ if $g(0) > \frac{R}{A} H_{0}$. Importantly, $I = 1$ implies that $\dot{g(t)} = r g(t) > 0$. As $t$ grows, the return for gross investment grows because the payoff period gets closer. $I = 1$ cannot be a solution forever because the agent receives no earnings if he invest all of the time during the complete life-cycle.\\
\indent The question, then, is if there is more than one period of specialization. We can use \eqref{eq:foctransversalitymodshen} and solve \eqref{eq:gmodshen} for any $I \in [0,1)$ to obtain
\begin{eqnarray}
g(t) = \frac{R(1-I)}{r} \left[ 1 - \exp^{r(t - T)} \right] \label{eq:gshendisc}
\end{eqnarray}

\noindent and note that $\dot{g} < 0$.\footnote{This needs to be completely rewritten in slide 22 - 24. Nothing makes sense. Simply copy all this. Conditions have mistakes in the HO.} Thus, there is at most one period of specialization and it happens at the beginning of the life cycle. This is a model of schooling.\\
\indent Finally, note that \eqref{eq:focinvestmentmodsehn} holds with strict equality at $t^*$ and \eqref{eq:humancapshenmod} is valid so that
\begin{equation}
g(t^*) = \frac{R}{A} \left( A t^* + H_{0} \right). 
\end{equation}

\noindent \eqref{eq:gshendisc} is also valid for $t^*$. Then,
\begin{equation}
\left( 1 - \exp^{r(t^*-T)} \right) = \frac{r}{A} \left( A t^* + H_{0} \right)
\end{equation}

\noindent and $\frac{\partial t^*}{\partial H_{0}} < 0, \frac{\partial t^*}{\partial A}> 0$ as in the model of Section \ref{section:baseline}. 
 
\subsection{Depreciation} \label{section:cycles}
Let us give some conditions under which human capital investment would have different episodes over the life cycle. First assume that $g(0) > \frac{H_{0} R}{A}$ so that there is an specialization period to begin with. We can solve \eqref{eq:focmotionmodshen} and \eqref{eq:gmodshen} to obtain
\begin{eqnarray}
H(t) &=& \left[ H_{0} - \frac{A}{\sigma} \right] \exp^{- \sigma t} + \frac{A}{\sigma} \nonumber \\
g(t) &=& g_{0} \exp^{(r + \sigma)t} \label{eq:gmodshent1} \nonumber
\end{eqnarray}

\noindent with $g_{0} > 0$. Once the solution becomes interior, $g(t) = \frac{R}{A} H(t)$ by \eqref{eq:focinvestmentmodsehn}. Assume that  $\sigma < \frac{A}{H_{0}}$ so that $\dot{H(0)} > 0$. Then, graphically,

\begin{center}
\begin{figure}[H]
\caption{Return to Gross Investment in Human Capital in the Modified Shenshinski Specification}
\centering
\includegraphics[width=4.5in, height=3.5in]{Figures/fig-shesh-for-intersection.pdf}
\floatfoot{\begin{small}
Note:
\end{small}}
\end{figure}
\end{center}

\indent Let $t_{1}$ denote the period in which the solution first period of specialization finishes. If the solutions ``bangs-out'' to $I=0$ we can use \eqref{eq:gmodshen} and the low of motion for human capital to obtain

\begin{eqnarray}
\dot{g} &=& (\sigma + r)g - R \nonumber \\
H(t) &=& H(t_{1}) \exp^{-\sigma(t - t_{1})} 
\end{eqnarray}

\noindent for $t_{1} < t < t_{2}$. Likewise, we can define a period $t_{2}$ in which the solutions ``bangs-in'' again and so on. Graphically,

\begin{center}
\begin{figure}[H]
\caption{Human Capital Investment Episodes in the Modified Shenshinski Specification} \label{figure:multi}
\centering
\includegraphics[width=4.5in, height=3.5in]{Figures/fig-shesh-one-traj.pdf}
\floatfoot{\begin{small}
Note:
\end{small}}
\end{figure}
\end{center}

\indent We informally build some conditions for this kind of cycling to happen. In $t < t_{1}$, $I = 1$ implies $\dot{g} > 0$. $g$ needs to decrease for the problem to respect the transversality condition. Thus, in the neighborhood of $t_{1}$ it has to be that $\dot{g(t_{1})} < \frac{R \dot{H(t_{1})}}{A}$ (see Figure \eqref{figure:multi}). If we take the expression from the right of $g(t_{1})$ this requires
\begin{eqnarray}
- R (\sigma + r) g(t_{1}) &<& \frac{R \dot{H(t_{1})}}{A} \nonumber \\
&=& \frac{- \sigma R H(t_{1})}{A} \nonumber \\
&=& -\sigma g(t_{1}) \nonumber \\
g(t_{1}) < \frac{R}{r}.  
\end{eqnarray}

\indent To have an initial period of specialization we need $g_{0} > \frac{R H_{0}}{A}$. At $t=0$, however, it should be the case that the slope of $\frac{R H_{0}}{A}$ exceeds $\dot{g}$. Otherwise the expressions for $g$ in the specialization period and the ``interior'' case do not intersect and the solution violates the transversality condition. This implies that $R \left[ 1 - \frac{\sigma H_{0}}{A} \right] > g_{0}(\sigma + r)$. High initial levels of human capital, low productivity, high discount, high depreciation, and low returns to human capital rule out an initial specialization period. Suppose the conditions described above hold so that specialization happens. We cannot show that $g(t_{3}) < g(t_{1})$ so that is better to accumulate ``all the human capital required for life'' in the first period of specialization. This is what we call cycling in the investment on human capital and is a consequence of depreciation.










