\section{Introduction}
\noindent Human capital investment and accumulation are essential components for many studies on a wide range of topics in Economics, including the economic growth and development of nations, the gender and black-white wage gaps, and the rate of return to schooling. \citet{becker1962investment} initiates the formal analysis of human capital studies and offers the first unified and comprehensive framework to study human capital investment with the standard Economic tools.\footnote{\citet{becker2009human} collates the work of the author on the subject which covers schooling, learning-by-doing, and on-the-job training.} \\
\indent Following Becker's work, \citet{ben1967production} proposes a dynamic model that relates human capital accumulation to life-cycle earnings. Currently, this is the workhorse model when it comes to analyze the relation between human capital accumulation decisions and life-cycle outcomes. The vast majority of papers that model human capital use some variation of the Ben-Porath model. These variations lead to very different time profiles of investment in human capital, human capital accumulation, and earnings.\\
\indent In this paper, we analyze the Ben-Porath model and several of its variations in order to inform researchers on the consequences of their modeling decisions. We proceed as follows: Section 2 presents the baseline specification, based on which the case with infinite time horizon and no depreciation is studied. By using the baseline model, many implications can be derived in closed form and the intuitions can be easily explained with straightforward graphical analysis. Section 3 analyzes the Haley-Rosen specification, which enables for finite horizon but still keeps tractability. Section 4 studies the model in a more general formulation by relaxing the neutrality assumption. Section 5 illustrates the case which generates a Bang-Bang equilibrium, meaning that an agent firstly makes full investment in human capital at the early part of life, which is followed by no investment at all after a certain point in time. In addition, Section 6 allows for depreciation and develops conditions under which investment in human capital happens in different episodes over the life-cycle. Finally, Section 7 offers some final comments. \\

