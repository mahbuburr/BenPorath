\noindent Human capital accumulation is an essential component of any model that aims to study some of the most relevant topics in Economics. Three representative examples are the economic growth and development of nations, the gender and black-white wage gaps, or the rate of return to schooling. \citet{becker1962investment} is the main predecessor in the analysis of investment in human capital. His work offers the first unified and comprehensive framework to study human capital investment with the usual tools of Economics.\footnote{\citet{becker2009human} collates the work of this author on the subject and summarizes his work on schooling, learning-by-doing, and on-the-job training.} \\
\indent Following this work, \citet{ben1967production} proposes a dynamic model that relates human capital accumulation to life-cycle earnings. Currently, this is the workhorse model when it comes to analyze human capital accumulation decision to life-cycle outcomes. The vast majority of papers that model human capital use some variation of the so-called Ben-Porath model (\textbf{see Figure 1}). Variations to the Ben-Porath model lead to very different time profiles of investment in human capital, human capital accumulation, and earnings.\\
\indent In this paper, we analyze the Ben-Porath model and several of its variations in order to inform researchers on the consequences of their modeling decisions when it comes to the relation between human capital accumulation and life-cycle outcomes. We proceed as follows: in Section 1 presents the baseline specification. Section 2 specializes the model to a case with no depreciation and infinite horizon. Many implications are easy to obtain in closed form and plot to gain intuition on the model. Section 3 analyzes what we call the Haley-Rosen, which enables for finite horizon but keeps tractability. Section 4 studies the model in its most general formulation. Section 5 allows for depreciation and develops conditions under which investment in human capital happens in different episodes over the life-cycle. Section 6 offers some final comments. \\

